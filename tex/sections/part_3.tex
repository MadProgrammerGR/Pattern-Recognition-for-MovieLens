\section{Ταξινόμηση χρηστών}

\subsection{Προεπεξεργασία των δεδομένων}
Τα δεδομένα που παρέχονται από το dataset για δοκιμές 5-fold είναι της μορφής [userid, movieid, rating, timestamp] και δεν είναι όπως τα χρειαζόμαστε.
Χρειαζόμαστε ένα τρόπο να αναπαριστούμε κάθε διάνυσμα ως στοιχεία του χρήστη και στοιχεία της ταινίας που βαθμολόγησε προκειμένου να δημιουργηθούν πρότυπα κλάσεων που μπορούμε να ταξινομήσουμε.
Γι’ αυτή τη δουλειά κάνουμε χρήση της python3 και του pandas library.
Σχηματίζουμε διανύσματα της μορφής [βαθμολογία, φύλο, ηλικία, ειδικότητα, κατηγορίες της ταινίας].
Συγκεκριμένα επειδή η ειδικότητα είναι ονομαστική μεταβλητή, δηλαδή διακριτών κατηγοριών, δε έχει νόημα να τη μετατρέψουμε σε ακέραιους ή πραγματικούς αριθμούς και έπειτα να τους συγκρίνουμε.
Αντί γι’ αυτό την μετατρέπουμε σε ένα 21-διάστατο διάνυσμα (όσες είναι οι δυνατές τιμές του) που έχει 1 στη στήλη που αντιστοιχεί στη τιμή του και 0 σε όλες τις άλλες (παρόμοια κωδικοποίηση με τα είδη της ταινίας).

Επίσης επειδή το τελικό ζητούμενο μας είναι να απαντήσουμε στο ερώτημα αν ο χρήστης είδε μια ταινία ή όχι, άρα πρέπει να χωρίσουμε με κάποιον τρόπο τα δεδομένα σε αυτές τις δυο κλάσεις (-1 και 1).
Κάνουμε την υπόθεση ότι οι βαθμολογίες που είναι μικρότερες ή ίσες του 3 σημαίνουν πως μάλλον ο χρήστης δεν έχει δει την ταινία, ενώ για βαθμολογίες μεγαλύτερες του 3 υποθέτουμε ότι την έχει δει.
Ίσως αυτή η υπόθεση να μην είναι η πιο κατάλληλη αλλά με βάση τα δεδομένα μας δεν έχουμε και πολύ μεγάλη ευελιξία στις επιλογές μας ώστε να τα διαχωρίσουμε σε δύο κλάσεις.

\subsection{Ταξινομητής ελαχίστων τετράγωνων (Least Squares)}
Ο ταξινομητής αυτός είναι ένας βέλτιστος γραμμικός ταξινομητής με την έννοια ότι ελαχιστοποιεί το άθροισμα των τετραγώνων σφαλμάτων μεταξύ της εκτίμησης της κλάσης κάθε διανύσματος με τη πραγματική κλάση που ανήκει.

Για να φτιάξουμε το καλύτερο ταξινομητή γι’ αυτά τα δεδομένα θα μπορούσαμε απλά να τον εκπαιδεύσουμε με όλο το σύνολο δεδομένων, όμως κάτι τέτοιο θα υπερπροσαρμόζοταν στα δεδομένα αυτά (overfitting) και θα έχανε τη δύναμη της γενικότητας (generality) που χρειάζεται για άγνωστα δεδομένα.
Γι’ αυτό το λόγο τον εκπαιδεύουμε 5 φορές στα ίδια δεδομένα όμως κάθε φορά με διαφορετικό σύνολο εκπαίδευσης και δοκιμής (5-fold validation), και τελικά διαλέγουμε από τους 5 ταξινομητές εκείνο με την καλύτερη απόδοση.

\subsection{Σχολιασμός αποτελεσμάτων του ταξινομητή}
Ύστερα από δοκιμές βλέπουμε ότι τα αποτελέσματα του ταξινομητή δεν είναι και πολύ ικανοποιητικά, καθώς όπως παρατηρούμε έχει επιτυχία περίπου 60\%.
Αυτό οφείλεται στο γεγονός ότι το σύνολο δεδομένων μας δεν είναι γραμμικώς διαχωρίσιμο και έτσι είναι αδύνατον ένας γραμμικός ταξινόμητης, όπως ο Least Squares, να τα ταξινομήσει με μεγάλη ακρίβεια.
Ένας πιο ικανός ταξινομητής γι’ αυτό το πρόβλημα θα ήταν ένας μη γραμμικός, όπως για παράδειγμα ένα νευρωνικό δίκτυο.
